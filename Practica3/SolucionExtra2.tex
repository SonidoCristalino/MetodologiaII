\documentclass[10pt, spanish]{article}

\usepackage{geometry} %Necesario para poder equiparar los m�rgenes
 \geometry{
    a4paper,
    total={170mm,257mm},
    left=20mm,
     top=20mm,
}
\usepackage[spanish]{babel}
\usepackage[latin1]{inputenc}

\usepackage{listings}
\usepackage{color}

\lstloadlanguages{C,C++,csh,Java}

\definecolor{red}{rgb}{0.6,0,0}
\definecolor{blue}{rgb}{0,0,0.6}
\definecolor{green}{rgb}{0,0.8,0}
\definecolor{cyan}{rgb}{0.0,0.6,0.6}
\definecolor{cloudwhite}{rgb}{0.9412, 0.9608, 0.8471}

\lstset{
    language=csh,
    basicstyle=\footnotesize\ttfamily,
    numbers=left,
    numberstyle=\tiny,
    numbersep=5pt,
    tabsize=2,
    extendedchars=true,
    breaklines=true,
    frame=b,
    stringstyle=\color{blue}\ttfamily,
    showspaces=false,
    showtabs=false,
    xleftmargin=17pt,
    framexleftmargin=17pt,
    framexrightmargin=5pt,
    framexbottommargin=4pt,
    commentstyle=\color{green},
    morecomment=[l]{//}, %use comment-line-style!
    morecomment=[s]{/*}{*/}, %for multiline comments
    showstringspaces=false,
    morekeywords={ abstract, event, new, struct,
    as, explicit, null, switch,
    base, extern, object, this,
    bool, false, operator, throw,
    break, finally, out, true,
    byte, fixed, override, try,
    case, float, params, typeof,
    catch, for, private, uint,
    char, foreach, protected, ulong,
    checked, goto, public, unchecked,
    class, if, readonly, unsafe,
    const, implicit, ref, ushort,
    continue, in, return, using,
    decimal, int, sbyte, virtual,
    default, interface, sealed, volatile,
    delegate, internal, short, void,
    do, is, sizeof, while,
    double, lock, stackalloc,
    else, long, static,
    enum, namespace, string},
    keywordstyle=\color{cyan},
    identifierstyle=\color{red},
    backgroundcolor=\color{cloudwhite},
}

\usepackage{caption}
\DeclareCaptionFont{white}{\color{white}}
\DeclareCaptionFormat{listing}{\colorbox{blue}{\parbox{\textwidth}{\hspace{15pt}#1#2#3}}}
\captionsetup[lstlisting]{format=listing,labelfont=white,textfont=white, singlelinecheck=false, margin=0pt, font={bf,footnotesize}}

\author{Emiliano Salvatori}
\title{Metodolog�a de la programaci�n II \\
    \large Practica III}
\date{Septiembre 2019}
\pagestyle{headings}

\begin{document}
\maketitle


% ----------------------------------------------------------------------
%                       INTRODUCION
% ----------------------------------------------------------------------
\section{Practica Extra III}

\subsection{Ejercitaci�n de mensajes SmallTakl}

Identificar la parte de cada mensaje y decir de qu� tipo es:
    \begin{enumerate}
        \item
            'casa' isNil.

            \textbf{Tipo de Mensaje}: Unario

            \textbf{Receptor}: 'casa'

            \textbf{Mensaje}: isNil

            \textbf{Selector tambi�n}: isNil
        \item
            9 + 3 * 2.

            \textbf{Tipo de Mensaje}: Varios mensajes Unarios
            \begin{itemize}

                \item
                \textbf{Primer mensaje}: de izquierda a derecha

                \item
                \textbf{Operaci�n}: 9 + 3

                \item
                \textbf{Receptor}: 9

                \item
                \textbf{Mensaje}: + 3

                \item
                \textbf{Selector }: +

                \item
                \textbf{Argumento }: 3

                \item
                \textbf{Valor de Retorno}: 12
            \end{itemize}

            \begin{itemize}
                \item
                    \textbf{Segundo mensaje}: de izquierda a derecha

                \item
                    \textbf{Operaci�n}: 12 * 2

                \item
                    \textbf{Receptor}: 12

                \item
                    \textbf{Mensaje}: * 2

                \item
                    \textbf{Selector }: *

                \item
                    \textbf{Argumento }: 2

                \item
                    \textbf{Valor de Retorno}: 24
            \end{itemize}


        \item
            true is false.

            \textbf{Tipo de Mensaje}: Binario

            \textbf{Receptor}: true

            \textbf{Mensaje}: is false

            \textbf{Selector }: is

            \textbf{Argumento}: False

        \item
            \#( 12 65 'olas' true) includes: 'viento'.

            \textbf{Tipo de Mensaje}: Mensaje de palabra clave, porque lleva '':''

            \textbf{Receptor}: \#( 12 65 'olas' true)

            \textbf{Selector }: include

            \textbf{Argumento}: viento

            \textbf{Mensaje}: includes 'viento'

            \textbf{Valor de Retorno}: false

        \item
            3 * 2 squared.

            \begin{itemize}

                \item
                \textbf{Primer mensaje}: de izquierda a derecha

                \item
                \textbf{Operaci�n}: 2 squared

                \item
                \textbf{Receptor}: 2

                \item
                \textbf{Mensaje}: squared

                \item
                \textbf{Selector }: squared

                \item
                \textbf{Argumento }: squared

                \item
                \textbf{Valor de Retorno}: 4
            \end{itemize}

            \begin{itemize}
                \item
                    \textbf{Segundo mensaje}: de izquierda a derecha

                \item
                    \textbf{Operaci�n}: 4 * 3

                \item
                    \textbf{Receptor}: 4

                \item
                    \textbf{Mensaje}: * 3

                \item
                    \textbf{Selector }: *

                \item
                    \textbf{Argumento }: 3

                \item
                    \textbf{Valor de Retorno}: 12
            \end{itemize}

        \item
            4 + 2 negated between: 3 + 4 * 5 and: 'hello' size * 10.

            \begin{itemize}

                \item
                \textbf{Primer mensaje}: de izquierda a derecha

                \item
                \textbf{Operaci�n}: 2 negated

                \item
                \textbf{Receptor}: 2

                \item
                \textbf{Mensaje}: negated

                \item
                \textbf{Selector }: negated

                \item
                \textbf{Argumento }: -

                \item
                \textbf{Valor de Retorno}: -2
            \end{itemize}

            \begin{itemize}
                \item
                    \textbf{Segundo mensaje}: de izquierda a derecha

                \item
                    \textbf{Operaci�n}: 4 + (-2)

                \item
                    \textbf{Receptor}: 4

                \item
                    \textbf{Mensaje}: - 2

                \item
                    \textbf{Selector }: -

                \item
                    \textbf{Argumento }: 2

                \item
                    \textbf{Valor de Retorno}: 2
            \end{itemize}

            \begin{itemize}
                \item
                    \textbf{Tercer mensaje}: de izquierda a derecha

                \item
                    \textbf{Operaci�n}: 3 + 4

                \item
                    \textbf{Receptor}: 3

                \item
                    \textbf{Mensaje}: + 4

                \item
                    \textbf{Selector }: +

                \item
                    \textbf{Argumento }: 4

                \item
                    \textbf{Valor de Retorno}: 7
            \end{itemize}

            \begin{itemize}
                \item
                    \textbf{Cuarto mensaje}: de izquierda a derecha

                \item
                    \textbf{Operaci�n}: 7 * 5

                \item
                    \textbf{Receptor}: 7

                \item
                    \textbf{Mensaje}: * 5

                \item
                    \textbf{Selector }: *

                \item
                    \textbf{Argumento }: 5

                \item
                    \textbf{Valor de Retorno}: 35
            \end{itemize}

            \begin{itemize}
                \item
                    \textbf{Quinto mensaje}: de izquierda a derecha. Mensaje unario

                \item
                    \textbf{Operaci�n}: 'hello' size

                \item
                    \textbf{Receptor}: 'hello'

                \item
                    \textbf{Mensaje}: size

                \item
                    \textbf{Selector }: size

                \item
                    \textbf{Argumento }: size

                \item
                    \textbf{Valor de Retorno}: 5
            \end{itemize}

            \begin{itemize}
                \item
                    \textbf{Sexto mensaje}: de izquierda a derecha. Mensaje Binario

                \item
                    \textbf{Operaci�n}: 5 * 10

                \item
                    \textbf{Receptor}: 5

                \item
                    \textbf{Mensaje}: * 10

                \item
                    \textbf{Selector }: *

                \item
                    \textbf{Argumento }: 10

                \item
                    \textbf{Valor de Retorno}: 50
            \end{itemize}

            \begin{itemize}
                \item
                    \textbf{S�ptimo mensaje}: de izquierda a derecha. Mensaje de Palabra Clave

                \item
                    \textbf{Operaci�n}: 2 between 35 and: 50

                \item
                    \textbf{Receptor}: 2

                \item
                    \textbf{Mensaje}: between 35 and: 50

                \item
                    \textbf{Selector }: between, and

                \item
                    \textbf{Argumento }: 35 y 50

                \item
                    \textbf{Valor de Retorno}: false
            \end{itemize}
    \end{enumerate}

\subsection{Ejercitaci�n de Diagramas de Clase y de Secuencia}

\subsubsection{Resolver el siguiente problema utilizando un Diagrama de Clase y luego en base a este, un diagrama de
Secuencia.}

\emph{Una veterinaria tiene informaci�n de los animales que atiende. De cada uno de
ellos se guarda una historia cl�nica; algunos animales deben seguir un
tratamiento.}

\emph{La veterinaria tiene un registro de clientes y guarda cada operaci�n monetaria
realizada por el mismo.}

\emph{La veterinaria adem�s de esto, vende productos, tiene un plan de vacunaci�n y
guarda la informaci�n de los animales que vacuna.}

\emph{La veterinaria tiene una agenda de turnos para la atenci�n m�dica de los
animales.}

\emph{Cada cliente tiene una cuenta en donde se registran los pagos realizados a la
veterinaria.}

\end{document}
