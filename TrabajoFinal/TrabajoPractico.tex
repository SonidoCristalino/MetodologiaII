
% ---------------------------------------------------------
%		Pre�mbulo del documento
% ---------------------------------------------------------
\documentclass[titlepage]{article} %utilizamos el documento de tipo articulo

\usepackage{comment} %util para comentarios largos
\usepackage[spanish]{babel}
\usepackage[latin1]{inputenc}
\usepackage{listings} %Necesario para colorear el c�digo
\usepackage{xcolor} %Necesario para obtener los colores para el codigo
\usepackage{libertine} %Necesario para la fuente
\usepackage[scaled=1.2]{inconsolata} %Necesario para la fuente
\usepackage{graphicx} % Necesario para poder renderizar imagenes
\usepackage[export]{adjustbox} %Necesarios para poder obtener el borde rojo de la imagen
\graphicspath{ {.} }

\definecolor{red}{rgb}{0.6,0,0}
\definecolor{blue}{rgb}{0,0,0.6}
\definecolor{green}{rgb}{0,0.5,0}
\definecolor{cyan}{rgb}{0.0,0.6,0.6}
\definecolor{cloudwhite}{rgb}{0.9412, 0.9608, 0.8471}

\lstset{
    language=c,
    basicstyle=\footnotesize\ttfamily,
    numbers=left,
    numberstyle=\tiny,
    numbersep=5pt,
    tabsize=1,
    extendedchars=true,
    breaklines=true,
    frame=b,
    stringstyle=\color{blue}\ttfamily,
    showspaces=false,
    showtabs=false,
    xleftmargin=17pt,
    framexleftmargin=17pt,
    framexrightmargin=5pt,
    framexbottommargin=4pt,
    commentstyle=\color{green},
    morecomment=[l]{//}, %use comment-line-style!
    morecomment=[s]{/*}{*/}, %for multiline comments
    showstringspaces=false,
    morekeywords={ abstract, event, new, struct,
    as, explicit, null, switch,
    base, extern, object, this,
    bool, false, operator, throw,
    break, finally, out, true,
    byte, fixed, override, try,
    case, float, params, typeof,
    catch, for, private, uint,
    char, foreach, protected, ulong,
    checked, goto, public, unchecked,
    class, if, readonly, unsafe,
    const, implicit, ref, ushort,
    continue, in, return, using,
    decimal, int, sbyte, virtual,
    default, interface, sealed, volatile,
    delegate, internal, short, void,
    do, is, sizeof, while,
    double, lock, stackalloc,
    else, long, static,
    enum, namespace, string},
    keywordstyle=\color{cyan},
    identifierstyle=\color{red},
    backgroundcolor=\color{cloudwhite},
}

\usepackage{caption}
\DeclareCaptionFont{white}{\color{white}}
\DeclareCaptionFormat{listing}{\colorbox{blue}{\parbox{\textwidth}{\hspace{15pt}#1#2#3}}}
\captionsetup[lstlisting]{format=listing,labelfont=white,textfont=white, singlelinecheck=false, margin=0pt, font={bf,footnotesize}}


% ---------------------------------------------------------
%		Pagina de Presentaci�n
% ---------------------------------------------------------

% Esto puede ser redefinido en un archivo aparte que contenga las directivas de \autor,\titulo,\subtitulo
% permitiendo de esta manera solamente incluir el archivo y las directivas. Ejemplo pagina 89 - Latex Cockbook

\author{Monasterio Salvatori Luparello Cantero}
\title{%
	Trabajo Pr�ctico\\
	\large Metodolog�a de la Programaci�n II}
\date{Octubre del 2019}
\pagestyle{headings} % para que indique seccion en cada pagina

% ---------------------------------------------------------
%		Comienza el docuemnto
% ---------------------------------------------------------

\begin{document}

% Se define cada tama�o de fuente para cada uno de las partes de la p�gina principal
\font\myfont=cmr12 at 30pt
\title{{\myfont Metodolog�a de la Programaci�n II\\
        \vspace{30pt}\Huge Trabajo Pr�ctico Final}}

%Se utiliza vspace para separa el Titulo de lo que vendr�a a ser el Subtitulo

\font\fuenteAutor=cmr12 at 20pt
\author{{\fuenteAutor Monasterio Salvatori Luparello Cantero}}

\font\fuenteFecha=cmr12 at 16pt
\date{{\fuenteFecha Octubre del 2019}}

\vbox{
	\centering
	\includegraphics[width=0.5\textwidth]{UNAJ.png}
	\maketitle
	}

\newpage
\tableofcontents % Directiva para que cree el listado de temas

\newpage

\section{Introducci�n}
En el siguiente informe se detalla lo realizado como parte del Trabajo Pr�ctico Final de la materia
\textbf{Metodolog�a de la Programaci�n II} para la \textbf{Comisi�n n� 2}.
El presente trabajo se basa en la planificaci�n y modelizaci�n de un negocio dedicado al alquiler de canchas de f�tbol.

Asimismo el presente trabajo vendr� acompa�ado para una mayor comprensi�n de los siguientes �tems:
\begin{itemize}
    \item
        Diagrama de Clases con sus correspondientes detalles y funcionalidades.
    \item
        Especificaci�n de la metodolog�a utilizada.
    \item
        Especificaci�n de los Patrones de Dise�o utilizados.
    \item
        C�digo refactorizado.
    \item
        Soportes te�ricos en los que se trabaj�.
    \item
        El c�digo fuente del trabajo realizado.
\end{itemize}

\section{Alcance del Negocio}

En un negocio llamado ''EstaD10s'' se alquilan 3 tipos de canchas: de 5, 7 y 11. Existen dos tipos de horarios: el
diurno y el nocturno. Este �ltimo aumenta en un 20\% en el costo de la cancha, por el gasto de luz. Cada turno se
reserva por una hora. Si quien alquila es socio, y tiene su cuota al d�a, obtiene un 30\% de descuento sobre el precio
final. La cuota social es de \$100. A continuaci�n se listan los tipos de canchas con sus respectivos precios:

\begin{itemize}
    \item
        \textbf{Cancha de 5}: Costo \$500
    \item
        \textbf{Cancha de 7}: Costo \$900
    \item
        \textbf{Cancha de 11}: Costo \$1500
\end{itemize}


\end{document}
